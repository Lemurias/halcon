\documentclass{beamer}
\usepackage[utf8]{inputenc}
\usepackage[spanish]{babel}
\usepackage{animate}
\usepackage{graphicx}
\usepackage{epstopdf}
\usepackage{listings}
\usepackage{caption}
\usepackage{subcaption}
\usepackage[sorting = none]{biblatex}
\usepackage{milstd}
\usepackage{stackengine}
\usepackage{scalerel}
\usepackage{wrapfig}
\usepackage{float}
\usepackage{setspace}
\newcommand\Bussymbol[1]{%
  \raisebox{-1.1em}{\scalebox{.7}{\stackunder{\BusWidth}{#1}}}%
}

\geometry{paperwidth=16cm,paperheight=9cm}

\epstopdfDeclareGraphicsRule{.gif}{png}{.png}{convert gif:#1 png:\OutputFile}
\AppendGraphicsExtensions{.gif}

\definecolor{inputNodeColor}{RGB}{255, 217, 102}
\definecolor{outputNodeColor}{RGB}{94, 165, 244}
\definecolor{clockNodeColor}{RGB}{255, 137, 137}

\usepackage{tikz}
\usetikzlibrary{shapes,arrows,fit,circuits.ee.IEC,positioning, decorations.pathreplacing, calc, arrows.meta}
\usepackage{verbatim}
\usepackage[siunitx]{circuitikz}
\usetikzlibrary{intersections}
\epstopdfDeclareGraphicsRule{.gif}{png}{.png}{convert gif:#1 png:\OutputFile}
\AppendGraphicsExtensions{.gif}
\usepackage{tkz-euclide}

\newsavebox{\imagebox}
%\usecolortheme{}

\lstset{language=C++, basicstyle=\tiny,
                keywordstyle=\color{blue},
                stringstyle=\color{red},
                commentstyle=\color{green},
                morecomment=[l][\color{magenta}]{\#}}


%% DIAGRAMA DE BLOQUE %%
\tikzstyle{block} = [draw, rectangle, 
    minimum height=3em, minimum width=6em]
\tikzstyle{large_sum} = [draw, rectangle, 
    minimum height=2em, minimum width=12em]
\tikzstyle{medium_sum_2} = [draw, rectangle, 
    minimum height=8em, minimum width=2em]
\tikzstyle{reg} = [draw, rectangle, 
    minimum height=2em, minimum width=0.25em, fill=black]
\tikzstyle{square_block} = [draw, rectangle, 
    minimum height=1em, minimum width=1em]
\tikzstyle{square_block_medium} = [draw, rectangle, 
    minimum height=3em, minimum width=3em]
\tikzstyle{sum} = [draw, circle, node distance=1cm]
\tikzstyle{node_input} = [draw, rectangle, minimum height=0.25em, minimum width=0.25em, fill=inputNodeColor]
\tikzstyle{node_output} = [draw, rectangle, minimum height=0.25em, minimum width=0.25em, fill=outputNodeColor]
\tikzstyle{node_clock} = [draw, rectangle, minimum height=0.25em, minimum width=0.25em, fill=clockNodeColor]

\addbibresource{references.bib}

\newcommand{\mypos}[2]{\tikz[remember picture]{\node[inner sep=0pt, anchor=base](#2){#1};}}

\title{HAWK}
\subtitle{Simuladores}
\institute{Fundación Fulgor}
\author[]{ Venancio, Mauro N.}
\date{\today}

\begin{document}
\def\radius{1.2 mm} 



\tikzset{slicerpic/.pic = {
\draw (-1,0) -- (1,0);
\draw (-0.7,-0.9)--(-0,-0.9) -- (0,0.9) --(0.7,0.9);
}}


\tikzstyle{vecArrow} = [thick, decoration={markings,mark=at position
   1 with {\arrow[semithick]{open triangle 60}}},
   double distance=1.4pt, shorten >= 5.5pt,
   preaction = {decorate},
   postaction = {draw,line width=1.4pt, white,shorten >= 4.5pt}]
\tikzstyle{innerWhite} = [semithick, white,line width=1.4pt, shorten >= 4.5pt]



\tikzset{clipper_pic/.pic = {
\draw (-1,-0.9)--(-0.5,-0.9) -- (0.5,0.9) --(1,0.9);
}}


\begin{frame}{TP7}
	\begin{figure}
		\centering
		\begin{tikzpicture}[thick,scale=0.6, every node/.style={scale=0.6}]
			\node[block, label={u\_symbol\_gen}](symbolgenerator){\begin{tabular}{c}
			Symbol \\
			Generator
			\end{tabular}};
			\node[square_block_medium, right of = symbolgenerator, node distance = 3 cm, label={u\_upsampling}] (upsampler){$\uparrow$};
			\node[block, right of = upsampler, node distance = 3.5 cm, label={u\_ser\_to\_par}] (u_ser_to_par){\begin{tabular}{c}
			Serial \\
			 to \\
			Parallel
			\end{tabular}};
			\node[block, right of = u_ser_to_par, node distance = 3.5 cm, label={u\_parallel\_filter}] (u_parallel_filter){\begin{tabular}{c}
			Parallel \\
			 FIR \\
			 Filter
			\end{tabular}};
			\node[block, right of = u_parallel_filter, node distance = 3.5 cm, label={u\_par\_to\_ser}] (u_par_to_ser){\begin{tabular}{c}
			Parallel \\
			 to \\
			Serial
			\end{tabular}};			
			\node[node_output, right of = u_par_to_ser, node distance = 3 cm, label = {0:o\_signal}] (o_signal) {} ;
			
			\draw[arrows=->, >=latex] (symbolgenerator) -- (upsampler);
			\draw[arrows=->, >=latex] (upsampler) -- (u_ser_to_par);
			
			
			
			
			\draw[arrows=->, >=latex] ($(u_ser_to_par.east) + (0, 0.6)$) -- ($(u_parallel_filter.west) + (0, 0.6)$);
			\draw[arrows=->, >=latex] ($(u_ser_to_par.east) + (0, 0.2)$) -- ($(u_parallel_filter.west) + (0, 0.2)$);
			\draw[arrows=->, >=latex] ($(u_ser_to_par.east) + (0, -0.2)$) -- ($(u_parallel_filter.west) + (0, -0.2)$);
			\draw[arrows=->, >=latex] ($(u_ser_to_par.east) + (0, -0.6)$) -- ($(u_parallel_filter.west) + (0, -0.6)$);
			
			\draw[arrows=->, >=latex] ($(u_parallel_filter.east) + (0, 0.6)$) -- ($(u_par_to_ser.west) + (0, 0.6)$);
			\draw[arrows=->, >=latex] ($(u_parallel_filter.east) + (0, 0.2)$) -- ($(u_par_to_ser.west) + (0, 0.2)$);
			\draw[arrows=->, >=latex] ($(u_parallel_filter.east) + (0, -0.2)$) -- ($(u_par_to_ser.west) + (0, -0.2)$);
			\draw[arrows=->, >=latex] ($(u_parallel_filter.east) + (0, -0.6)$) -- ($(u_par_to_ser.west) + (0, -0.6)$);



			\node[node_clock, label = {180:clk\_slow}] at ($ (symbolgenerator) - (2, 1.5) $) (i_clock){};
			\node[node_clock, below of = i_clock, node distance = 1 cm, label={180:clk\_fast}] (i_clock_os){};
			
			\draw[dotted, arrows=->, >=latex] (i_clock) -| (symbolgenerator);
			\draw[dotted, arrows=->, >=latex] (i_clock_os) -| (upsampler);
			\node [right of = i_clock, node distance = 5cm] (aux_clk_slow) {}; 
			
			\draw[dotted, arrows=-, >=latex] (i_clock) -- ($(aux_clk_slow.center) + (-0.25,0)$);
			\draw[dotted, arrows=->, >=latex] ($(aux_clk_slow.center) + (+0.25,0)$) -| ($(u_ser_to_par.south) + (-0.5,0)$);
			\draw[dotted, arrows=->, >=latex] (i_clock_os) -| ($(u_ser_to_par.south) + (0.5,0)$);
			
			\node [right of = i_clock, node distance = 9cm] (aux_clk_slow_2) {}; 
			\draw[dotted, arrows=-, >=latex] ($(aux_clk_slow.center) + (+0.25,0)$) -- ($(aux_clk_slow_2.center) + (-0.25,0)$);
			\draw[dotted, arrows=->, >=latex] ($(aux_clk_slow_2.center) + (+0.25,0)$) -| (u_parallel_filter.south);
			\draw[dotted, arrows=->, >=latex] ($(aux_clk_slow_2.center) + (+0.25,0)$) -| ($(u_par_to_ser.south) + (-0.5,0)$);
			\draw[dotted, arrows=->, >=latex] (i_clock_os) -| ($(u_par_to_ser.south) + (0.5,0)$);
			\draw[arrows=->, >=latex] (u_par_to_ser) -- (o_signal);
			
			% Module
            \draw [ rounded corners,dashed,postaction={draw,line width=1pt}](i_clock_os) -- ++(0,-0.75) -| (o_signal) -- ++(0,1.75) -| (i_clock) -- (i_clock_os);
		\end{tikzpicture}
		\caption{TP7 Connection}
		\label{fig:transmisor}
	\end{figure}

\end{frame}

%%%%%%%%%%%%%%%%%%%%%%%%%%%%%%%%%%%%%%%%%%%%%%%%%%%%%%%%%%%

\begin{frame}{Parallel FIR Filter}

	\begin{figure}
		\centering
		\begin{tikzpicture}[thick,scale=0.5, every node/.style={scale=0.5}]

			\node[node_input, label={180:i\_signal[3]}, label={15:x\_t[3]}] (i_signal_3){};
			\node[node_input, below of = i_signal_3, label={180:i\_signal[2]}, label={15:x\_t[2]}, node distance = 1cm] (i_signal_2){};
			\node[node_input, below of = i_signal_2, label={180:i\_signal[1]}, label={15:x\_t[1]}, node distance = 1cm] (i_signal_1){};
			\node[node_input, below of = i_signal_1, label={180:i\_signal[0]}, label={15:x\_t[0]}, node distance = 1cm] (i_signal_0){};
			
			\node[square_block, below right of = i_signal_0, node distance = 2.5 cm, label={180:\tiny coeffs[0]}] (h_0_0){\tiny h[0]};
			%\node[below right of = i_signal_0, node distance = 2 cm, label={0:\tiny h(0)}] (h_0_0){};
			\node[square_block, right of = h_0_0, node distance = 1 cm] (h_0_1){\tiny h[0]};
			\node[square_block, right of = h_0_1, node distance = 1 cm] (h_0_2){\tiny h[0]};
			\node[square_block, right of = h_0_2, node distance = 1 cm] (h_0_3){\tiny h[0]};
			
			\node[square_block, right of = h_0_3, node distance = 2 cm, label={180:\tiny coeffs[1]}] (h_1_0){\tiny h[1]};
			\node[square_block, right of = h_1_0, node distance = 1 cm] (h_1_1){\tiny h[1]};
			\node[square_block, right of = h_1_1, node distance = 1 cm] (h_1_2){\tiny h[1]};
			\node[square_block, right of = h_1_2, node distance = 1 cm] (h_1_3){\tiny h[1]};
			
			\node[square_block, right of = h_1_3, node distance = 2 cm, label={180:\tiny coeffs[2]}] (h_2_0){\tiny h[2]};
			\node[square_block, right of = h_2_0, node distance = 1 cm] (h_2_1){\tiny h[2]};
			\node[square_block, right of = h_2_1, node distance = 1 cm] (h_2_2){\tiny h[2]};
			\node[square_block, right of = h_2_2, node distance = 1 cm] (h_2_3){\tiny h[2]};
			
			\node[square_block, right of = h_2_3, node distance = 2 cm, label={180:\tiny coeffs[3]}] (h_3_0){\tiny h[3]};
			\node[square_block, right of = h_3_0, node distance = 1 cm] (h_3_1){\tiny h[3]};
			\node[square_block, right of = h_3_1, node distance = 1 cm] (h_3_2){\tiny h[3]};
			\node[square_block, right of = h_3_2, node distance = 1 cm] (h_3_3){\tiny h[3]};
			
			
			
			
			
			%%%%%%%%%%%%%%%%%%%%%%%%%%%%
			
			%\node[reg, right of = i_signal_2, node distance = 16 cm, label={r\_reg[0]}] (r_reg_0) {};
			%%\draw[arrows=->,>=latex] (i_signal_2) -- (r_reg_0);
			%\draw[arrows=->,>=latex, name path=line 1] (i_signal_2) -- (r_reg_0);
			%\path[name path=line 2] (i_signal_3) -| (h_2_2);
			%\path [name intersections={of = line 1 and line 2}];
  			%\coordinate (S)  at (intersection-1);
  			%\path[name path=circle] (S) circle(\radius);
  			%\path[name intersections={of = circle and line 2}];
  			%\coordinate (I1)  at (intersection-1);
  			%\coordinate (I2)  at (intersection-2);
  			%\draw (i_signal_3) -| ($ (I1) + (0,-0.03) $);
  			%\draw[arrows=->,>=latex] ($ (I2) + (0,+0.03) $) -- (h_2_2);
			%\tkzDrawArc[color=black, line width=1pt](S,I1)(I2);
			
			%%%%%%%%%%%%%%%%%%%%%%%%%%%%%			
			
			\node[reg, right of = i_signal_2, node distance = 19.2 cm, label={r\_reg[2]}] (r_reg_2) {};
			\draw[arrows=->,>=latex] (i_signal_2) -- (r_reg_2);
			
			\node[reg, right of = i_signal_1, node distance = 14.2 cm, label={r\_reg[1]}] (r_reg_1) {};
			\draw[arrows=->,>=latex, orange!40!gray] (i_signal_1) -- (r_reg_1);
			
			\node[reg, right of = i_signal_0, node distance = 9.45 cm, label={r\_reg[0]}] (r_reg_0) {};
			\draw[arrows=->,>=latex, purple!40!gray] (i_signal_0) -- (r_reg_0);			
			
			
			\draw[arrows=->,>=latex, blue!40!gray] (i_signal_3) -| (h_0_3);
			\draw[arrows=->,>=latex] (i_signal_2) -| (h_0_2);
			\draw[arrows=->,>=latex, orange!40!gray] (i_signal_1) -| (h_0_1);
			\draw[arrows=->,>=latex, purple!40!gray] (i_signal_0) -| (h_0_0);
			
			\draw[arrows=->,>=latex, red!40!gray] (r_reg_0) -| (h_1_3);
			\draw[arrows=->,>=latex, blue!40!gray] (i_signal_3) -| (h_1_2);
			\draw[arrows=->,>=latex] (i_signal_2) -| (h_1_1);
			\draw[arrows=->,>=latex, orange!40!gray] (i_signal_1) -| (h_1_0);
			
			\draw[arrows=->,>=latex, green!40!gray] (r_reg_1) -| (h_2_3);
			\draw[arrows=->,>=latex, red!40!gray] (r_reg_0) -| (h_2_2);
			\draw[arrows=->,>=latex,  blue!40!gray] (i_signal_3) -| (h_2_1);
			\draw[arrows=->,>=latex] (i_signal_2) -| (h_2_0);
			
			\draw[arrows=->,>=latex, cyan!40!gray] (r_reg_2) -| (h_3_3);
			\draw[arrows=->,>=latex, green!40!gray] (r_reg_1) -| (h_3_2);
			\draw[arrows=->,>=latex, red!40!gray] (r_reg_0) -| (h_3_1);
			\draw[arrows=->,>=latex,  blue!40!gray] (i_signal_3) -| (h_3_0);
			
				
			%\node[large_sum, below right of = h_0_0, node distance = 2 cm, label={0:\tiny h(0)}] (sum_0) ++ (0,5) {};
			\node[large_sum] at ($ (h_0_0) + (1.5,-6.4) $) (sum_0) {+};
			\node[large_sum] at ($ (h_1_0) + (1.5,-6.4) $) (sum_1) {+};
			\node[large_sum] at ($ (h_2_0) + (1.5,-6.4) $) (sum_2) {+};
			\node[large_sum] at ($ (h_3_0) + (1.5,-6.4) $) (sum_3) {+};
		
			
	
	
			
			\draw[arrows=->,>=latex, blue!40!gray] (h_0_3) -- ++ (0,-6);
			
			\draw[arrows=-,>=latex] (h_0_2) |- ++ (6,-2);			
			\draw[arrows=->,>=latex] ($(h_0_2) + (6,-2)$) -- ++ (0,-4);
			
			\draw[arrows=-,>=latex, orange!40!gray] (h_0_1) |- ++ (12,-3);
			\draw[arrows=->,>=latex, orange!40!gray] ($(h_0_1) + (12,-3)$) -- ++ (0,-3);

			\draw[arrows=-,>=latex, purple!40!gray] (h_0_0) |- ++ (18,-3.5);
			\draw[arrows=->,>=latex, purple!40!gray] ($(h_0_0) + (18,-3.5)$) -- ++ (0,-2.5);
			
			
			
			\draw[arrows=-,>=latex, red!40!gray] (h_1_3) |- ++ (-2.5,-1.5);
			\draw[arrows=-,>=latex, red!40!gray] ($(h_1_3) + (-2.5,-1.5)$) -- ++ (0,-2.5);
			\draw[arrows=->,>=latex, red!40!gray] ($(h_1_3) + (-2.5,-4)$) -| ++ (-3.5,-2);
			
			\draw[arrows=->,>=latex, blue!40!gray] (h_1_2) -- ++ (0,-6);
			
			\draw[arrows=-,>=latex] (h_1_1) |- ++ (3,-1);
			\draw[arrows=-,>=latex] ($(h_1_1) + (3,-1)$) -- ++ (0,-1);
			\draw[arrows=->,>=latex] ($(h_1_1) + (3,-2)$) -| ++ (3,-4);			
		
			\draw[arrows=-,>=latex, orange!40!gray] (h_1_0) |- ++ (12,-2.5);
			\draw[arrows=->,>=latex, orange!40!gray] ($(h_1_0) + (12,-2.5)$) -- ++ (0,-3.5);
			
			
			
			\draw[arrows=-,>=latex, green!40!gray] (h_2_3) |- ++ (-2.5,-1.5);
			\draw[arrows=-,>=latex, green!40!gray] ($(h_2_3) + (-2.5,-1.5)$) -- ++ (0,-3);
			\draw[arrows=->,>=latex, green!40!gray] ($(h_2_3) + (-2.5,-4.5)$) -| ++ (-9.5,-1.5);
			
			\draw[arrows=-,>=latex, red!40!gray] (h_2_2) |- ++ (-2.5,-0.5);
			\draw[arrows=-,>=latex, red!40!gray] ($(h_2_2) + (-2.5,-0.5)$) -- ++ (0,-3.5);
			\draw[arrows=->,>=latex, red!40!gray] ($(h_2_2) + (-2.5,-4)$) -| ++ (-3.5,-2);
			
			\draw[arrows=->,>=latex,  blue!40!gray] (h_2_1) -- ++ (0,-6);
			
			\draw[arrows=-,>=latex] (h_2_0) |- ++ (6,-1);
			\draw[arrows=->,>=latex] ($(h_2_0) + (6,-1)$) -- ++ (0,-5);
			
			
			
			\draw[arrows=-,>=latex, cyan!40!gray] (h_3_3) |- ++ (-4,-3);
			\draw[arrows=-,>=latex, cyan!40!gray] ($(h_3_3) + (-4,-3)$) -| ++ (0,-2.5);
			\draw[arrows=->,>=latex, cyan!40!gray] ($(h_3_3) + (-4,-5.5)$) -| ++ (-14,-0.5);
			
			\draw[arrows=-,>=latex, green!40!gray] (h_3_2) |- ++ (-2.5,-1.5);
			\draw[arrows=-,>=latex, green!40!gray] ($(h_3_2) + (-2.5,-1.5)$) -- ++ (0,-3.5);
			\draw[arrows=->,>=latex, green!40!gray] ($(h_3_2) + (-2.5,-5)$) -| ++ (-9.5,-1);

			\draw[arrows=-,>=latex, red!40!gray] (h_3_1) |- ++ (-2.5,-0.5);
			\draw[arrows=-,>=latex, red!40!gray] ($(h_3_1) + (-2.5,-0.5)$) -- ++ (0,-3.5);
			\draw[arrows=->,>=latex, red!40!gray] ($(h_3_1) + (-2.5,-4)$) -| ++ (-3.5,-2);

			\draw[arrows=->,>=latex,  blue!40!gray] (h_3_0) -- ++ (0,-6);
			
			
			
			
			
			
			
			\node[node_output, right of = i_signal_3, node distance = 24.5 cm, label={0:o\_signal[3]}, label={165:y\_t[3]}] (o_signal_3){};
			\node[] at ($ (sum_0) + (1,-1.5) $) (sum_aux_0){};
			\node[] at ($ (o_signal_3) + (-1.5,0) $) (o_signal_aux_3){};
			\draw[arrows=-,>=latex,  blue!40!gray] (sum_0.south) |- (sum_aux_0.center);
			\draw[arrows=-,>=latex,  blue!40!gray] (sum_aux_0.center) -| (o_signal_aux_3.center);
			\draw[arrows=->,>=latex,  blue!40!gray] (o_signal_aux_3.center) -- (o_signal_3);
			
			\node[node_output, below of = o_signal_3, node distance = 1.2 cm, label={0:o\_signal[2]}, label={165:y\_t[2]}] (o_signal_2){};
			\node[] at ($ (sum_1) + (1,-1.2) $) (sum_aux_1){};
			\node[] at ($ (o_signal_2) + (-2,0) $) (o_signal_aux_2){};
			\draw[arrows=-,>=latex] (sum_1.south) |- (sum_aux_1.center);
			\draw[arrows=-,>=latex] (sum_aux_1.center) -| (o_signal_aux_2.center);
			\draw[arrows=->,>=latex] (o_signal_aux_2.center) -- (o_signal_2);
			
			\node[node_output, below of = o_signal_2, node distance = 1.2 cm, label={0:o\_signal[1]}, label={165:y\_t[1]}] (o_signal_1){};
			\node[] at ($ (sum_2) + (1,-0.9) $) (sum_aux_2){};
			\node[] at ($ (o_signal_1) + (-2.5,0) $) (o_signal_aux_1){};
			\draw[arrows=-,>=latex, purple!40!gray] (sum_2.south) |- (sum_aux_2.center);
			\draw[arrows=-,>=latex, purple!40!gray] (sum_aux_2.center) -| (o_signal_aux_1.center);
			\draw[arrows=->,>=latex, purple!40!gray] (o_signal_aux_1.center) -- (o_signal_1);
			
			\node[node_output, below of = o_signal_1, node distance = 1.2 cm, label={0:o\_signal[0]}, label={165:y\_t[0]}] (o_signal_0){};
			\node[] at ($ (sum_3) + (1,-0.6) $) (sum_aux_3){};
			\node[] at ($ (o_signal_0) + (-3,0) $) (o_signal_aux_0){};
			\draw[arrows=-,>=latex, orange!40!gray] (sum_3.south) |- (sum_aux_3.center);
			\draw[arrows=-,>=latex, orange!40!gray] (sum_aux_3.center) -| (o_signal_aux_0.center);
			\draw[arrows=->,>=latex, orange!40!gray] (o_signal_aux_0.center) -- (o_signal_0);



			
			
			
			
			
			\node[node_clock, below of = i_signal_0, node distance = 7 cm, label={180:i\_clock}] (clock){};
			\node[] at ($ (clock) + (0.25,6) $) (node_aux_0) {};
			\draw[dotted,arrows=-,>=latex] (clock) -| (node_aux_0.center);
			
            \draw[dotted,arrows=->,>=latex] (node_aux_0.center) -| (r_reg_2.south);
            \draw[dotted,arrows=->,>=latex] ($(r_reg_1) + (0,-2)$) -| (r_reg_1.south);
            \draw[dotted,arrows=->,>=latex] ($(r_reg_0) + (0,-1)$) -| (r_reg_0.south);
            
            % Module
            \draw [ rounded corners,dashed,postaction={draw,line width=1pt}](clock) -- ++(0,-2.85) -| (o_signal_0) -- (o_signal_1)-- (o_signal_2) -- (o_signal_3) -- ++(0,1.5) -| (i_signal_3) -- (i_signal_3) -- (i_signal_2) -- (i_signal_1) -- (i_signal_0) -- (clock);
			
		
		\end{tikzpicture}
		\label{fig:modulo_lms}
	\end{figure}
	
\end{frame}


%%%%%%%%%%%%%%%%%%%%%%%%%%%%%%%%%%%%%%%%%%%%%%%%%%%%%%%%%%%


\begin{frame}{Equations}

\doublespacing
		$y\_t[0] = x\_t[0] \cdot coeffs[0] + ~~~~~~~ x\_t[1] \cdot coeffs[1] + ~~~~~~~ x\_t[2] \cdot coeffs[2] + ~~~~~~~ x\_t[3] \cdot coeffs[3]$
		\\
		$y\_t[1] = x\_t[1] \cdot coeffs[0] + ~~~~~~~ x\_t[2] \cdot coeffs[1] + ~~~~~~~ x\_t[3] \cdot coeffs[2] + r\_reg.o[0] \cdot coeffs[3]$
		\\
		$y\_t[2] = x\_t[2] \cdot coeffs[0] + ~~~~~~~ x\_t[3] \cdot coeffs[1] + r\_reg.o[0] \cdot coeffs[2] + r\_reg.o[1] \cdot coeffs[3]$
		\\
		$y\_t[3] = x\_t[3] \cdot coeffs[0] + r\_reg.o[0] \cdot coeffs[1] + r\_reg.o[1] \cdot coeffs[2] + r\_reg.o[2] \cdot coeffs[3]$

	
\end{frame}






\end{document}
